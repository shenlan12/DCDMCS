\defmodule{DigitalSequence}

This abstract class describes methods specific to digital sequences.
Concrete classes must implement the \method{extendSequence}{} method
that increases the number of points of the digital sequence.
Calling the  methods \method{toNet}{} or \method{toNetShiftCj}{}
will transform the digital sequence into a digital net, which has
a fixed number of points $n$.

%%%%%%%%%%%%%%%%%%%%%%%%%%%%%%%%%%%%%%%%
\bigskip\hrule\bigskip

\begin{code}
\begin{hide}
/*
 * Class:        DigitalSequence
 * Description:  abstract class with methods specific to digital sequences
 * Environment:  Java
 * Software:     SSJ 
 * Copyright (C) 2001  Pierre L'Ecuyer and Universite de Montreal
 * Organization: DIRO, Universite de Montreal
 * @author       
 * @since

 * SSJ is free software: you can redistribute it and/or modify it under
 * the terms of the GNU General Public License (GPL) as published by the
 * Free Software Foundation, either version 3 of the License, or
 * any later version.

 * SSJ is distributed in the hope that it will be useful,
 * but WITHOUT ANY WARRANTY; without even the implied warranty of
 * MERCHANTABILITY or FITNESS FOR A PARTICULAR PURPOSE.  See the
 * GNU General Public License for more details.

 * A copy of the GNU General Public License is available at
   <a href="http://www.gnu.org/licenses">GPL licence site</a>.
 */
\end{hide}
package umontreal.iro.lecuyer.hups;


public abstract class DigitalSequence extends DigitalNet \begin{hide} { 

\end{hide}

   public abstract void extendSequence (int k);
\end{code} 
\begin{tabb}
    Increases the number of points to $n = b^k$ from now on.
\end{tabb}
\begin{htmlonly}
   \param{k}{there will be b^k points}
\end{htmlonly}
\begin{code}\begin{hide}

   private int[][] copyDigitalShift (int[][] S) {
      // Copy the shift S into T and returns T.
      if (S == null) return null;
      int[][] T = new int [S.length][S[0].length];
      for (int i = 0; i < S.length; ++i)
         for (int j = 0; j < S[0].length; ++j)
            T[i][j] = S[i][j];
      return T;
   }

   private DigitalNet initNetVar (boolean shiftFlag) {
      // Initializes the net for the two toNet methods below.
      DigitalNet net = new DigitalNet ();
      if (shiftFlag)
         net.dim = dim + 1;
      else
         net.dim = dim;
      net.numPoints = numPoints;
      net.numCols = numCols;
      net.numRows = numRows;
      net.outDigits = outDigits;
      net.normFactor = normFactor;
      net.b = b;
      net.factor = new double[outDigits];
      for (int i = 0; i < outDigits; i++)
         net.factor[i] = factor[i];
      net.genMat = new int[net.dim * numCols][numRows];
      net.shiftStream = shiftStream;
      net.capacityShift = capacityShift;
      net.dimShift = dimShift;
      net.digitalShift = copyDigitalShift (digitalShift);
      if (shiftFlag && shiftStream != null) {
         net.addRandomShift (dimShift, dimShift + 1, shiftStream);
      }
      return net;
   } \end{hide}

   public DigitalNet toNet() \begin{hide} {
      DigitalNet net = initNetVar (false);
      final int N = dim * numCols;
      for (int i = 0; i < N; i++)
         for (int j = 0; j < numRows; j++)
            net.genMat[i][j] = genMat[i][j];
      return net;
   } \end{hide}
\end{code} 
\begin{tabb}
   Transforms this digital sequence into a digital net without changing
   the coordinates of the points. Returns the digital net.
\end{tabb}
\begin{code}

   public DigitalNet toNetShiftCj() \begin{hide} {
      DigitalNet net = initNetVar (true);
      int j, c, l, start;

      /* Shift all coordinates from the sequence by 1 dimension */
      for (j = dim; j >= 1; j--) {
         start = j * numCols;
         for (c = 0; c < numCols; c++)
            for (l = 0; l < numRows; l++) 
               net.genMat[start+c][l] = genMat[start - numCols + c][l];
      }

      // j = 0: initialize C_0 to the reflected identity.
      for (c = 0; c < numCols; c++) {
         for (l = 0; l < numRows; l++)
            net.genMat[c][l] = 0;
         net.genMat[c][numCols-c-1] = 1;
      }
      return net;
  }\end{hide}
\end{code} 
\begin{tabb}
  Transforms this digital sequence into a digital net by adding one dimension
  and shifting all coordinates by one position. The first coordinate of point
  $i$ is $i/n$, where $n$ is the total number of points.
  Thus if the coordinates of a point of the digital sequence were
  $(x_0, x_1, x_2, \ldots, x_{s-1})$, then the coordinates of the
  point of the digital net will be $(i/n, x_0, x_1, \ldots, x_{s-1})$.
  In other words, for the digital net, $\mathbf{C}_0$ is the reflected 
  identity and for $j\ge 1$, the $\mathbf{C}_j$ used is the
  $\mathbf{C}_{j-1}$ of the digital sequence. If the digital sequence uses
  a digital shift, then the digital net will include the digital shift with
  one more dimension also.  Returns the digital net.
\end{tabb}
\begin{code}

   public PointSetIterator iteratorShift()\begin{hide} {
      return new DigitalNetIteratorShiftGenerators();
   }\end{hide}
\end{code}
\begin{tabb}
  Similar to \method{iterator}{}, except that the first coordinate
  of the points is $i/n$, the second coordinate is obtained via
  the generating matrix $\mathbf{C}_0$, the next one via $\mathbf{C}_1$, 
  and so on. Thus, this iterator shifts all coordinates of each point
  one position to the right and sets the first coordinate of point $i$
  to  $i/n$, so that the points enumerated with this iterator have one more
  dimension. A digital shift, if present, will have one more dimension also.
  This iterator uses the Gray code.
\end{tabb}
\begin{code}

   public PointSetIterator iteratorShiftNoGray()\begin{hide} {
      return new DigitalNetIteratorShiftNoGray();
   }\end{hide}
\end{code}
\begin{tabb}
  This iterator shifts all coordinates of each point one position to the right
  and sets the first coordinate of point $i$  to  $i/n$, so that the points 
  enumerated with this iterator have one more dimension. 
  This iterator does not use the Gray code; the points are enumerated in the 
  order of their first coordinate before randomization.
  A digital shift, if present, will have one more dimension also.
\end{tabb}
\begin{code}\begin{hide}

// ************************************************************************

   protected class DigitalNetIteratorShiftGenerators 
                   extends DigitalNetIterator {
      // Similar to DigitalNetIterator; the first coordinate
      // of point i is i/n, and all the others are shifted one position
      // to the right. The points have dimension = dim + 1.

      public DigitalNetIteratorShiftGenerators() {
         super();
         dimS = dim + 1;
         if (digitalShift != null && dimShift < dimS)
            addRandomShift (dimShift, dimS, shiftStream);
         init2 ();
      }

      public void init() {   // This method is necessary to overload
      }                      // the init() of DigitalNetIterator 

      public void init2() { // See constructor
         resetCurPointIndex();
      }


      public void setCurPointIndex (int i) {
         if (i == 0) {
            resetCurPointIndex();
            return;
         }
         curPointIndex = i;
         curCoordIndex = 0;

         // Digits of Gray code, used to reconstruct cachedCurPoint.
         idigits = intToDigitsGray (b, i, numCols, bdigit, gdigit);
         int c, j, l, sum;
         for (j = 1; j <= dim; j++) {
            for (l = 0; l < outDigits; l++) {
               if (digitalShift == null) 
                  sum = 0;
               else 
                  sum = digitalShift[j][l]; 
               if (l < numRows) 
                  for (c = 0; c < idigits; c++)
                     sum += genMat[(j - 1)*numCols+c][l] * gdigit[c];
               cachedCurPoint [j*outDigits+l] = sum % b;
            }
         }
         // The case j = 0
         for (l = 0; l < outDigits; l++) {
            if (digitalShift == null) 
               sum = 0;
            else 
               sum = digitalShift[0][l]; 
            if (l < numRows) 
               for (c = 0; c < idigits; c++)
                  if (l == numCols-c-1)
                     sum += gdigit[c];
            cachedCurPoint [l] = sum % b;
         }         
      }

      public int resetToNextPoint() {
         // incremental computation.
         curPointIndex++;
         curCoordIndex = 0;
         if (curPointIndex >= numPoints)
            return curPointIndex;

         // Update the digital expansion of i in base b, and find the 
         // position of change in the Gray code. Set all digits == b-1 to 0
         // and increase the first one after by 1.
         int pos;      // Position of change in the Gray code.
         for (pos = 0; gdigit[pos] == b-1; pos++)  
            gdigit[pos] = 0;
         gdigit[pos]++;

         // Update the cachedCurPoint by adding the column of the gener. 
         // matrix that corresponds to the Gray code digit that has changed.
         // The digital shift is already incorporated in the cached point.
         int c, j, l;
         int lsup = numRows;        // Max index l
         if (outDigits < numRows)
            lsup = outDigits;
         for (j = 1; j <= dim; j++) {
            for (l = 0; l < lsup; l++) {
               cachedCurPoint[j*outDigits + l] +=
                   genMat[(j-1)*numCols + pos][l];
               cachedCurPoint[j * outDigits + l] %= b;
            }
         }
         // The case j = 0
         l = numCols-pos-1;
         if (l < lsup) {
            cachedCurPoint[l] += 1;
            cachedCurPoint[l] %= b;
         }

         return curPointIndex;
      }
   }


// ************************************************************************

   protected class DigitalNetIteratorShiftNoGray 
                   extends DigitalNetIterator {
      // Similar to DigitalNetIterator; the first coordinate
      // of point i is i/n, and all the others are shifted one position
      // to the right. The points have dimension = dim + 1.

      public DigitalNetIteratorShiftNoGray() {
         super();
         dimS = dim + 1;
         if (digitalShift != null && dimShift < dimS)
            addRandomShift (dimShift, dimS, shiftStream);
         init2();
      }

      public void init() {   // This method is necessary to overload
      }                      // the init() of DigitalNetIterator 

      public void init2() { // See constructor
         resetCurPointIndex();
      }

      public void setCurPointIndex (int i) {
         if (i == 0) {
            resetCurPointIndex();
            return;
         }
         curPointIndex = i;
         curCoordIndex = 0;

         // Convert i to b-ary representation, put digits in bdigit.
         idigits = intToDigitsGray (b, i, numCols, bdigit, gdigit);
         int c, j, l, sum;
         for (j = 1; j <= dim; j++) {
            for (l = 0; l < outDigits; l++) {
               if (digitalShift == null) 
                  sum = 0;
               else 
                  sum = digitalShift[j][l]; 
               if (l < numRows) 
                  for (c = 0; c < idigits; c++) {
                     sum += genMat[(j-1)*numCols+c][l] * bdigit[c];
                     sum %= b;
                  }
               cachedCurPoint [j*outDigits+l] = sum;
            }
         }
         // The case j = 0
         for (l = 0; l < outDigits; l++) {
            if (digitalShift == null) 
               sum = 0;
            else 
               sum = digitalShift[0][l]; 
            if (l < numRows) 
               for (c = 0; c < idigits; c++)
                  if (l == numCols-c-1)
                     sum += bdigit[c];
            cachedCurPoint [l] = sum % b;
         }
      }

      public int resetToNextPoint() {
         curPointIndex++;
         curCoordIndex = 0;
         if (curPointIndex >= numPoints)
            return curPointIndex;

         // Find the position of change in the digits of curPointIndex in base
         // b. Set all digits = b-1 to 0; increase the first one after by 1.
         int pos;
         for (pos = 0; bdigit[pos] == b-1; pos++)  
            bdigit[pos] = 0;
         bdigit[pos]++;

         // Update the digital expansion of curPointIndex in base b.
         // Update the cachedCurPoint by adding 1 unit at the digit pos.
         // If pos > 0, remove b-1 units in the positions < pos. Since 
         // calculations are mod b, this is equivalent to adding 1 unit.
         // The digital shift is already incorporated in the cached point.
         int c, j, l;
         int lsup = numRows;        // Max index l
         if (outDigits < numRows)
            lsup = outDigits;
         for (j = 1; j <= dim; j++) {
            for (l = 0; l < lsup; l++) {
               for (c = 0; c <= pos; c++)
                  cachedCurPoint[j*outDigits + l] +=
                     genMat[(j-1)*numCols + c][l];
               cachedCurPoint[j * outDigits + l] %= b;
            }
         }
         // The case j = 0
         for (l = 0; l < lsup; l++) {
            for (c = 0; c <= pos; c++)
               if (l == numCols-c-1) {
                  cachedCurPoint[l] += 1;
                  cachedCurPoint[l] %= b;
               }
         }

         return curPointIndex;
      }

   }
}
\end{hide}
\end{code}
