\defmodule {ContinuousDistribution}

Classes implementing continuous distributions should inherit from this base
 class.
Such distributions are characterized by a \emph{density} function $f(x)$,
thus the signature of a \texttt{density} method is supplied here.
This class also provides default implementations for $\bar F(x)$
and for $F^{-1}(u)$, the latter using the Brent-Dekker method to find the inverse
of a generic distribution function $F$. % The integer \texttt{decPrec} defines
% the target number of decimals of precision when approximating a distribution
% function, but there is \emph{no guarantee} that this target is always attained.

%% It specifies the signature of methods for computing the density
%% $f(x)$
%% , the distribution function $F(x)=P[X\le x]$, the
%% complementary distribution function $\bar F(x)$,
%% and the inverse distribution function $ F^{-1} (u)$
%% for a random variable $X$ having a univariate continuous distribution.

\bigskip\hrule

\begin{code}
\begin{hide}
/*
 * Class:        ContinuousDistribution
 * Description:  continuous distributions
 * Environment:  Java
 * Software:     SSJ
 * Copyright (C) 2001  Pierre L'Ecuyer and Universite de Montreal
 * Organization: DIRO, Universite de Montreal
 * @author
 * @since

 * SSJ is free software: you can redistribute it and/or modify it under
 * the terms of the GNU General Public License (GPL) as published by the
 * Free Software Foundation, either version 3 of the License, or
 * any later version.

 * SSJ is distributed in the hope that it will be useful,
 * but WITHOUT ANY WARRANTY; without even the implied warranty of
 * MERCHANTABILITY or FITNESS FOR A PARTICULAR PURPOSE.  See the
 * GNU General Public License for more details.

 * A copy of the GNU General Public License is available at
   <a href="http://www.gnu.org/licenses">GPL licence site</a>.
 */
\end{hide}
package umontreal.iro.lecuyer.probdist;
\begin{hide}
import umontreal.iro.lecuyer.util.PrintfFormat;
import umontreal.iro.lecuyer.util.Num;
import umontreal.iro.lecuyer.functions.MathFunction;
\end{hide}

public abstract class ContinuousDistribution implements Distribution \begin{hide} {
   @Deprecated
   public int decPrec = 15;

   private int getDecPrec() {
      return decPrec;
   }\end{hide}
\end{code}
%% \begin{tabb}  A rough target on the number of decimals of precision when
%%   approximating a distribution function.
%%   (There is \emph{no guarantee} that this target is always attained.)
%% \end{tabb}
\begin{code}\begin{hide}
    // x infinity for some distributions
    protected static final double XBIG = 100.0;
    protected static final double XBIGM = 1000.0;

    // [supportA, supportB] is the support of the pdf(x)
    protected double supportA = Double.NEGATIVE_INFINITY;
    protected double supportB = Double.POSITIVE_INFINITY;

    // EPSARRAY[j]: Epsilon required for j decimal degits of precision
    protected static final double[] EPSARRAY = {
    0.5, 0.5E-1, 0.5E-2, 0.5E-3, 0.5E-4, 0.5E-5, 0.5E-6, 0.5E-7, 0.5E-8,
    0.5E-9, 0.5E-10, 0.5E-11, 0.5E-12, 0.5E-13, 0.5E-14, 0.5E-15, 0.5E-16,
    0.5E-17, 0.5E-18, 0.5E-19, 0.5E-20, 0.5E-21, 0.5E-22, 0.5E-23, 0.5E-24,
    0.5E-25, 0.5E-26, 0.5E-27, 0.5E-28, 0.5E-29, 0.5E-30, 0.5E-31, 0.5E-32,
    0.5E-33, 0.5E-34, 0.5E-35
    };\end{hide}

   public abstract double density (double x);
\end{code}
\begin{tabb} Returns $f(x)$, the density evaluated at $x$.
\end{tabb}
\begin{htmlonly}
   \param{x}{value at which the density is evaluated}
   \return{density function evaluated at \texttt{x}}
\end{htmlonly}
% \begin{code}
%
%    public MathFunction density()\begin{hide} {
%       return new MathFunction() {
%          public double evaluate(double x) {
%             return density(x);
%          }
%       };
%    }\end{hide}
% \end{code}
% \begin{tabb} Returns an object allowing the computation of the density
%   at any point.
% \end{tabb}
% \begin{htmlonly}
%    \return{density object
% \end{htmlonly}
\begin{code}

   public double barF (double x)\begin{hide} {
      return 1.0 - cdf (x);
   }\end{hide}
\end{code}
 \begin{tabb} Returns the complementary distribution function.
    The default implementation computes $\bar F(x) = 1 - F(x)$.
    %, which is not accurate when $F(x)\to 1$.
 \end{tabb}
\begin{htmlonly}
   \param{x}{value at which the complementary distribution function is evaluated}
   \return{complementary distribution function evaluated at \texttt{x}}
\end{htmlonly}
\begin{code}\begin{hide}

   private void findInterval (double u, double [] iv) {
      // Finds an interval [a, b] that certainly contains x defined as
      // u = cdf(x). The result is written in iv[0] = a and iv[1] = b.

      if (u > 1.0 || u < 0.0)
         throw new IllegalArgumentException ("u not in [0, 1]");
      final double XLIM =  Double.MAX_VALUE/2.0;
      final double B0 = 8.0;
      double b = B0;
      while (b < XLIM && u > cdf(b))
         b *= 2.0;
      if (b > B0) {
         iv[0] = b/2.0;
         iv[1] = Math.min (b, supportB);
         return;
      }

      double a = -B0;
      while (a > -XLIM && u < cdf(a))
         a *= 2.0;
      if (a < -B0) {
         iv[1] = a/2.0;
         iv[0] = Math.max (a, supportA);
         return;
      }
      iv[0] = Math.max (a, supportA);
      iv[1] = Math.min (b, supportB);
   }\end{hide}

   public double inverseBrent (double a, double b, double u, double tol) \begin{hide} {
      if (u > 1.0 || u < 0.0)
          throw new IllegalArgumentException ("u not in [0, 1]");
      if (b < a) {
         double ctemp = a;   a = b;   b = ctemp;
      }
      if (u <= 0.0) {
          System.out.println ("********** WARNING,  inverseBrent:   u = 0");
          return supportA;
      }
      if (u >= 1.0) {
          System.out.println ("********** WARNING,  inverseBrent:   u = 1");
          return supportB;
      }
      final int MAXITER = 50;      // Maximum number of iterations
      tol += EPSARRAY[decPrec] + Num.DBL_EPSILON;    // in case tol is too small
      double ua = cdf(a) - u;
      if (ua > 0.0)
          throw new IllegalArgumentException ("u < cdf(a)");
      double ub = cdf(b) - u;
      if (ub < 0.0)
          throw new IllegalArgumentException ("u > cdf(b)");

      final boolean DEBUG = false;
      if (DEBUG) {
          String ls = System.getProperty("line.separator");
          System.out.println (
             "-------------------------------------------------------------"
              + ls + "u = " + PrintfFormat.g (20, 15, u));
          System.out.println
           (ls + "iter           b                  c               F(x) - u" + ls);
      }
      // Initialize
      double c = a;
      double uc = ua;
      double len = b - a;
      double t = len;
      if (Math.abs(uc) < Math.abs(ub)) {
            a = b; b = c; c = a;
            ua = ub; ub = uc; uc = ua;
      }
      int i;
      for (i = 0; i < MAXITER; ++i) {
         double tol1 = tol + 4.0*Num.DBL_EPSILON*Math.abs(b);
         double xm = 0.5*(c - b);
         if (DEBUG) {
            System.out.println (PrintfFormat.d (3, i) + "  " +
                PrintfFormat.g (18, decPrec, b) + "  " +
                PrintfFormat.g (18, decPrec, c) + "  " +
                PrintfFormat.g (14, 4, ub));
         }
         if (Math.abs(ub) == 0.0 || (Math.abs(xm) <= tol1)) {
            if (b <= supportA) return supportA;
            if (b >= supportB) return supportB;
            return b;
         }

         double s, p, q, r;
         if ((Math.abs(t) >= tol1) && (Math.abs(ua) > Math.abs(ub))) {
            if (a == c) {
               // linear interpolation
               s = ub/ua;
               q = 1.0 - s;
               p = 2.0 * xm * s;
            } else {
               // quadratic interpolation
               q = ua / uc;
               r = ub / uc;
               s = ub / ua;
               p = s*(2.0*xm*q*(q - r) - (b - a)*(r - 1.0));
               q = (q - 1.0)*(r - 1.0)* (s - 1.0);
            }
            if (p > 0.0)
               q = -q;
            p = Math.abs(p);

            // Accept interpolation?
            if ((2.0*p >= (3.0*xm*q - Math.abs(q*tol1))) ||
                (p >= Math.abs(0.5*t*q))) {
               len = xm;
               t = len;
            } else {
               t = len;
               len = p/q;
            }

         } else {
            len = xm;
            t = len;
         }

         a = b;
         ua = ub;
         if (Math.abs(len) > tol1)
            b += len;
         else if (xm < 0.0)
            b -= tol1;
         else
            b += tol1;
         ub = cdf(b) - u;

         if (ub*(uc/Math.abs(uc)) > 0.0) {
            c = a;
            uc = ua;
            len = b - a;
            t = len;
         } else if (Math.abs(uc) < Math.abs(ub)) {
            a = b; b = c; c = a;
            ua = ub; ub = uc; uc = ua;
         }
      }
      if (i >= MAXITER) {
         String lineSep = System.getProperty("line.separator");
         System.out.println (lineSep +
           "*********** inverseBrent:   no convergence after " + MAXITER +
           " iterations");
      }
      return b;
   }\end{hide}
\end{code}
\begin{tabb} Computes the inverse distribution function $x = F^{-1}(u)$,
    using the Brent-Dekker method. The interval $[a, b]$ \emph{must} contain
    the root $x$ such that $F(a) \le u \le F(b)$, where $u=F(x)$.
    The calculations are done with an approximate precision of \texttt{tol}.
    Returns $x = F^{-1}(u)$. Restrictions: $u \in [0,1]$.
\end{tabb}
\begin{htmlonly}
   \param{a}{left endpoint of initial interval}
   \param{b}{right endpoint of initial interval}
   \param{u}{value at which the inverse distribution function is evaluated}
   \param{tol}{accuracy goal}
   \return{inverse distribution function evaluated at \texttt{u}}
\end{htmlonly}
\begin{code}

   public double inverseBisection (double u)\begin{hide} {
      final int MAXITER = 100;              // Maximum number of iterations
      final double EPSILON = EPSARRAY[decPrec];  // Absolute precision
      final double XLIM =  Double.MAX_VALUE/2.0;
      final boolean DEBUG = false;
      final String lineSep = System.getProperty("line.separator");

      if (u > 1.0 || u < 0.0)
          throw new IllegalArgumentException ("u not in [0, 1]");
      if (decPrec > Num.DBL_DIG)
          throw new IllegalArgumentException ("decPrec too large");
      if (decPrec <= 0)
          throw new IllegalArgumentException ("decPrec <= 0");
      if (DEBUG) {
          System.out.println ("---------------------------" +
              " -----------------------------" + lineSep +
               PrintfFormat.f (10, 8, u));
      }

      double x = 0.0;
      if (u <= 0.0) {
          x = supportA;
          if (DEBUG) {
             System.out.println (lineSep + "            x                   y" +
                 lineSep + PrintfFormat.g (17, 2, x) + " " +
                 PrintfFormat.f (17, decPrec, u));
          }
          return x;
      }
      if (u >= 1.0) {
          x = supportB;
          if (DEBUG) {
             System.out.println (lineSep + "            x                   y" +
                            lineSep + PrintfFormat.g (17, 2, x) + " " +
                            PrintfFormat.f (17, decPrec, u));
          }
          return x;
      }

      double [] iv = new double [2];
      findInterval (u, iv);
      double xa = iv[0];
      double xb = iv[1];
      double yb = cdf(xb) - u;
      double ya = cdf(xa) - u;
      double y;

      if (DEBUG)
          System.out.println (lineSep +
             "iter              xa                   xb           F - u");

      boolean fini = false;
      int i = 0;
      while (!fini) {
          if (DEBUG)
             System.out.println (PrintfFormat.d (3, i) + "  " +
                                 PrintfFormat.g (18, decPrec, xa) + "  " +
                                 PrintfFormat.g (18, decPrec, xb) + "  " +
                                 PrintfFormat.g (14, 4, y));
          x = (xa + xb)/2.0;
          y = cdf(x) - u;
          if ((y == 0.0) ||
              (Math.abs ((xb - xa)/(x + Num.DBL_EPSILON)) <= EPSILON))  {
              fini = true;
              if (DEBUG)
                System.out.println (lineSep + "                x" +
                     "                     u" + lineSep +
                     PrintfFormat.g (20, decPrec, x) + "  " +
                     PrintfFormat.g (18, decPrec, y+u));
          }
          else if (y*ya < 0.0)
             xb = x;
          else
             xa = x;
          ++i;

          if (i > MAXITER) {
                //System.out.println (lineSep +
                //  "** inverseF:SEARCH DOES NOT SEEM TO CONVERGE");
              fini = true;
          }
      }
      return x;
   }\end{hide}
\end{code}
\begin{tabb}
    Computes and returns the inverse distribution function $x = F^{-1}(u)$,
%    $F^{-1}(u) = \inf\{x\in\RR : F(x)\ge u\}$,
    using bisection. Restrictions: $u \in [0,1]$.
\end{tabb}
\begin{htmlonly}
   \param{u}{value at which the inverse distribution function is evaluated}
   \return{the inverse distribution function evaluated at \texttt{u}}
   \exception{IllegalArgumentException}{if $u$ is  not in the interval $[0,1]$}
\end{htmlonly}
\begin{code}

   public double inverseF (double u)\begin{hide} {
      double [] iv = new double [2];
      findInterval (u, iv);
      return inverseBrent (iv[0], iv[1], u, EPSARRAY[decPrec]);
   }\end{hide}
\end{code}
\begin{tabb}
    Returns the inverse distribution function $x = F^{-1}(u)$.
 %   by calling the method \method{inverseBrent}{}.
 %   $F^{-1}(u) = \inf\{x\in\RR : F(x)\ge u\}$,
    Restrictions: $u \in [0,1]$.
\end{tabb}
\begin{htmlonly}
   \param{u}{value at which the inverse distribution function is evaluated}
   \return{the inverse distribution function evaluated at \texttt{u}}
   \exception{IllegalArgumentException}{if $u$ is  not in the interval $[0,1]$}
\end{htmlonly}
\begin{code}

   public double getMean()\begin{hide} {
      throw new UnsupportedOperationException("getMean is not implemented ");
   }\end{hide}
\end{code}
\begin{tabb} Returns the mean.
\end{tabb}
\begin{htmlonly}
   \return{the mean}
\end{htmlonly}
\begin{code}

   public double getVariance()\begin{hide} {
      throw new UnsupportedOperationException("getVariance is not implemented ");
   }\end{hide}
\end{code}
\begin{tabb} Returns the variance.
\end{tabb}
\begin{htmlonly}
   \return{the variance}
\end{htmlonly}
\begin{code}

   public double getStandardDeviation()\begin{hide} {
      throw new UnsupportedOperationException (
         "getStandardDeviation is not implemented ");
   }
\end{hide}
\end{code}
\begin{tabb} Returns the standard deviation.
\end{tabb}
\begin{htmlonly}
   \return{the standard deviation}
\end{htmlonly}
\begin{code}

   public double getXinf()\begin{hide} {
      return supportA;
   }\end{hide}
\end{code}
\begin{tabb} Returns $x_a$ such that the probability density is 0 everywhere
 outside the interval $[x_a, x_b]$.
\end{tabb}
\begin{htmlonly}
   \return{lower limit of support}
\end{htmlonly}
\begin{code}

   public double getXsup()\begin{hide} {
      return supportB;
   }
\end{hide}
\end{code}
\begin{tabb} Returns $x_b$ such that the probability density is 0 everywhere
 outside the interval $[x_a, x_b]$.
\end{tabb}
\begin{htmlonly}
   \return{upper limit of support}
\end{htmlonly}
\begin{code}

   public void setXinf (double xa)\begin{hide} {
      supportA = xa;
   }\end{hide}
\end{code}
\begin{tabb} Sets the value $x_a=$ \texttt{xa}, such that the probability
 density is 0 everywhere outside the interval $[x_a, x_b]$.
\end{tabb}
\begin{htmlonly}
   \param{xa}{lower limit of support}
\end{htmlonly}
\begin{code}

   public void setXsup (double xb)\begin{hide} {
      supportB = xb;
   }
}\end{hide}
\end{code}
\begin{tabb} Sets the value $x_b=$ \texttt{xb}, such that the probability
 density is 0 everywhere outside the interval $[x_a, x_b]$.
\end{tabb}
\begin{htmlonly}
   \param{xb}{upper limit of support}
\end{htmlonly}
