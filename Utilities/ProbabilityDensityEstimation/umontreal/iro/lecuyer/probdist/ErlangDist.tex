\defmodule {ErlangDist}

Extends the class \class{GammaDist} for the special case
of the {\em Erlang\/} distribution with
shape parameter $k > 0$ and scale parameter $\lambda > 0$.
This distribution is a special case of the gamma distribution
for which the shape parameter $k=\alpha$ is an integer.

\bigskip\hrule

%%%%%%%%%%%%%%%%%%%%%%%%%%%%%%%%%%%%%%%%
\begin{code}
\begin{hide}
/*
 * Class:        ErlangDist
 * Description:  Erlang distribution
 * Environment:  Java
 * Software:     SSJ
 * Copyright (C) 2001  Pierre L'Ecuyer and Universite de Montreal
 * Organization: DIRO, Universite de Montreal
 * @author
 * @since

 * SSJ is free software: you can redistribute it and/or modify it under
 * the terms of the GNU General Public License (GPL) as published by the
 * Free Software Foundation, either version 3 of the License, or
 * any later version.

 * SSJ is distributed in the hope that it will be useful,
 * but WITHOUT ANY WARRANTY; without even the implied warranty of
 * MERCHANTABILITY or FITNESS FOR A PARTICULAR PURPOSE.  See the
 * GNU General Public License for more details.

 * A copy of the GNU General Public License is available at
   <a href="http://www.gnu.org/licenses">GPL licence site</a>.
 */
\end{hide}
package umontreal.iro.lecuyer.probdist;

public class ErlangDist extends GammaDist\begin{hide} {
\end{hide}\end{code}
%%%%%%%%%%%%%%%%%%%%%%%%%%%%%%%%%%%%%%%%%
\subsubsection* {Constructors}

\begin{code}

   public ErlangDist (int k)\begin{hide} {
      super (k);
   }\end{hide}
\end{code}
  \begin{tabb} Constructs a \texttt{ErlangDist} object with parameters
  $k$ = \texttt{k} and $\lambda=1$.
  \end{tabb}
\begin{code}

   public ErlangDist (int k, double lambda)\begin{hide} {
      super (k, lambda);
   }\end{hide}
\end{code}
  \begin{tabb} Constructs a \texttt{ErlangDist} object with parameters
  $k$ = \texttt{k}  and $\lambda$ = \texttt{lambda}.
  \end{tabb}

%%%%%%%%%%%%%%%%%%%%%%%%%%%%%%%%%%%
\subsubsection* {Methods}

\begin{code}

   public static double density (int k, double lambda, double x)\begin{hide} {
      return density ((double)k, lambda, x);
   }\end{hide}
\end{code}
\begin{tabb} Computes the density function.
\end{tabb}
\begin{code}

   public static double cdf (int k, double lambda, int d, double x)\begin{hide} {
      return cdf ((double)k, d, lambda*x);
   }\end{hide}
\end{code}
 \begin{tabb} Computes the distribution function using
   the gamma distribution function.
\end{tabb}
\begin{code}

   public static double barF (int k, double lambda, int d, double x)\begin{hide} {
      return barF ((double)k, d, lambda*x);
   }\end{hide}
\end{code}
\begin{tabb} Computes the complementary distribution function.
\end{tabb}
\begin{code}

   public static double inverseF (int k, double lambda, int d, double u)\begin{hide} {
      return inverseF ((double)k, lambda, d, u);
   }\end{hide}
\end{code}
\begin{tabb}  Returns the inverse distribution function.
\end{tabb}
\begin{code}

   public static double[] getMLE (double[] x, int n)\begin{hide} {
      double parameters[] = GammaDist.getMLE (x, n);
      parameters[0] = Math.round (parameters[0]);
      return parameters;
   }\end{hide}
\end{code}
\begin{tabb}
  Estimates the parameters $(k,\lambda)$ of the Erlang distribution
   using the maximum likelihood method, from the $n$ observations
   $x[i]$, $i = 0, 1,\ldots, n-1$. The estimates are returned in a two-element
    array, in regular order: [$k$, $\lambda$].
   \begin{detailed}
   The equations of the maximum likelihood are the same as for
   the gamma distribution. The $k$ parameter is the rounded value of the
   $\alpha$ parameter of the gamma distribution, and the $\lambda$ parameter
   is equal to the $\beta$ parameter of the gamma distribution.
   \end{detailed}
\end{tabb}
\begin{htmlonly}
   \param{x}{the list of observations used to evaluate parameters}
   \param{n}{the number of observations used to evaluate parameters}
   \return{returns the parameters [$\hat{k}$, $\hat{\lambda}$]}
\end{htmlonly}
\begin{code}

   public static ErlangDist getInstanceFromMLE (double[] x, int n)\begin{hide} {
      double parameters[] = getMLE (x, n);
      return new ErlangDist ((int) parameters[0], parameters[1]);
   }\end{hide}
\end{code}
\begin{tabb}
   Creates a new instance of an Erlang distribution with parameters $k$ and
   $\lambda$ estimated
   using the maximum likelihood method based on the $n$ observations $x[i]$,
   $i = 0, 1, \ldots, n-1$.
\end{tabb}
\begin{htmlonly}
   \param{x}{the list of observations to use to evaluate parameters}
   \param{n}{the number of observations to use to evaluate parameters}
\end{htmlonly}
\begin{code}

   public static double getMean (int k, double lambda)\begin{hide} {
      if (k <= 0)
         throw new IllegalArgumentException ("k <= 0");
      if (lambda <= 0.0)
         throw new IllegalArgumentException ("lambda <= 0");
      return (k / lambda);
   }\end{hide}
\end{code}
\begin{tabb}  Computes and returns the mean, $E[X] = k/\lambda$,
   of the Erlang distribution with parameters $k$ and $\lambda$.
\end{tabb}
\begin{htmlonly}
   \return{the mean of the Erlang distribution $E[X] = k / \lambda$}
\end{htmlonly}
\begin{code}

   public static double getVariance (int k, double lambda)\begin{hide} {
      if (k <= 0)
         throw new IllegalArgumentException ("k <= 0");
      if (lambda <= 0.0)
         throw new IllegalArgumentException ("lambda <= 0");

      return (k / (lambda * lambda));
   }\end{hide}
\end{code}
\begin{tabb}  Computes and returns the variance, $\mbox{Var}[X] = k/\lambda^2$,
   of the Erlang distribution with parameters $k$ and $\lambda$.
\end{tabb}
\begin{htmlonly}
   \return{the variance of the Erlang distribution $\mbox{Var}[X] = k / \lambda^2$}
\end{htmlonly}
\begin{code}

   public static double getStandardDeviation (int k, double lambda)\begin{hide} {
      if (k <= 0)
         throw new IllegalArgumentException ("k <= 0");
      if (lambda <= 0.0)
         throw new IllegalArgumentException ("lambda <= 0");

      return (Math.sqrt (k) / lambda);
   }\end{hide}
\end{code}
\begin{tabb}  Computes and returns the standard deviation of the Erlang
   distribution with parameters $k$ and $\lambda$.
\end{tabb}
\begin{htmlonly}
   \return{the standard deviation of the Erlang distribution}
\end{htmlonly}
\begin{code}

   public int getK()\begin{hide} {
      return (int) getAlpha();
   }\end{hide}
\end{code}
  \begin{tabb} Returns the parameter $k$ for this object.
  \end{tabb}
\begin{code}

   public void setParams (int k, double lambda, int d)\begin{hide} {
      super.setParams (k, lambda, d);
   }\end{hide}
\end{code}
  \begin{tabb} Sets the parameters $k$ and $\lambda$ of the distribution for this
  object.  Non-static methods are computed with a rough target of
  \texttt{d} decimal digits of precision.
  \end{tabb}
\begin{code}

   public double[] getParams ()\begin{hide} {
      return super.getParams();
   }\end{hide}
\end{code}
\begin{tabb}
   Return a table containing parameters of the current distribution.
   This table is put in regular order: [$k$, $\lambda$].
\end{tabb}
\begin{hide}\begin{code}

   public String toString ()\begin{hide} {
      return getClass().getSimpleName() + " : k = " + (int)super.getAlpha() + ", lambda = " + super.getLambda();
   }\end{hide}
\end{code}
\begin{tabb}
   Returns a \texttt{String} containing information about the current distribution.
\end{tabb}\end{hide}
\begin{code}\begin{hide}
}\end{hide}
\end{code}
