\defmodule {HypoExponentialDistEqual}

This class implements  the \emph{hypoexponential} distribution for the case of equidistant $\lambda_{i} = (n+1-i)h$. We have $\lambda_{i+1} - \lambda_i =
h$, with $h$  a constant, and $n \ge k$ are integers.

\noindent The formula (\ref{eq:convolution-hypo}) becomes
\begin{equation}
\begin{latexonly}%
\bar F(x) = \mathbb{P}\left[X_1 + \cdots + X_k > x \right]
  = \sum_{i=1}^k e^{-(n+1-i)h x} \prod_{\substack{j=1\\ j\not=i}}^k
  \frac{n+1-j}{i - j}.
\label{eq:conv-hypo-equal}
\end{latexonly}%
\begin{htmlonly}%
\bar F(x) = \mathcal{P}\left[X_1 + \cdots + X_k > x \right]
  = \sum_{i=1}^k e^{-(n+1-i)h x} \prod_{\substack{j=1\\ j\not=i}}^k
  {n+1-j}/{i - j}.
\end{htmlonly}%
\end{equation}
%
The formula (\ref{eq:fhypoexp2}) for the density becomes
\begin{equation}
\begin{latexonly}%
  f(x) = \sum_{i=1}^k (n+1-i)h e^{-(n+1-i)h x} \prod_{\substack{j=1\\ j\not=i}}^k
  \frac{n+1-j}{i - j}.
\eqlabel{eq:fhypoexp3}
\end{latexonly}%
\begin{htmlonly}%
 f(x) =\sum_{i=1}^k (n+1-i)h e^{-(n+1-i)h x}
   \prod_{\substack{j=1\\ j\not=i}}^k {n+1-j}/{i - j}.
\end{htmlonly}%
\end{equation}

\bigskip\hrule

%%%%%%%%%%%%%%%%%%%%%%%%%%%%%%%%%%%%%%%%
\begin{code}\begin{hide}
/*
 * Class:        HypoExponentialDistEqual
 * Description:  Hypo-exponential distribution
 * Environment:  Java
 * Software:     SSJ
 * Copyright (C) 2001  Pierre L'Ecuyer and Universite de Montreal
 * Organization: DIRO, Universite de Montreal
 * @author       Richard Simard
 * @since        February 2014

 * SSJ is free software: you can redistribute it and/or modify it under
 * the terms of the GNU General Public License (GPL) as published by the
 * Free Software Foundation, either version 3 of the License, or
 * any later version.

 * SSJ is distributed in the hope that it will be useful,
 * but WITHOUT ANY WARRANTY; without even the implied warranty of
 * MERCHANTABILITY or FITNESS FOR A PARTICULAR PURPOSE.  See the
 * GNU General Public License for more details.

 * A copy of the GNU General Public License is available at
   <a href="http://www.gnu.org/licenses">GPL licence site</a>.
 */
\end{hide}
package umontreal.iro.lecuyer.probdist;
\begin{hide}
import java.util.Formatter;
import java.util.Locale;
import umontreal.iro.lecuyer.util.*;
import umontreal.iro.lecuyer.functions.MathFunction;\end{hide}

public class HypoExponentialDistEqual extends HypoExponentialDist\begin{hide} {
   private double m_h;
   private int m_n;
   private int m_k;
\end{hide}
\end{code}
%%%%%%%%%%%%%%%%%%%%%%%%%%%%%%%%%%%%%%%%%
\subsubsection* {Constructor}

\begin{code}

   public HypoExponentialDistEqual (int n, int k, double h)\begin{hide} {
      super (null);
      setParams (n, k, h);
   }\end{hide}
\end{code}
  \begin{tabb} Constructor for equidistant rates.
  The rates are $\lambda_i = (n+1-i)h$, for $i = 1,\ldots,k$.
  \end{tabb}
\begin{htmlonly}
   \param{n}{largest rate is $nh$}
   \param{k}{number of rates}
   \param{h}{difference between adjacent rates}
\end{htmlonly}


%%%%%%%%%%%%%%%%%%%%%%%%%%%%%%%%%%%
\subsubsection* {Methods}

\begin{code}\begin{hide}
   public double density (double x) {
      return density (m_n, m_k, m_h, x);
   }

   public double cdf (double x) {
      return cdf (m_n, m_k, m_h, x);
   }

   public double barF (double x) {
      return barF (m_n, m_k, m_h, x);
   }

   public double inverseF (double u) {
      return inverseF (m_n, m_k, m_h, u);
   }\end{hide}

   public static double density (int n, int k, double h, double x)\begin{hide} {
      if (x < 0)
         return 0;
      double r = -Math.expm1(-h*x);
      double v = BetaDist.density(k, n - k + 1, r);
      return h*v*Math.exp(-h*x);
   }\end{hide}
\end{code}
  \begin{tabb} Computes the density function $f(x)$, with the same arguments
  as in the constructor.
\end{tabb}
\begin{htmlonly}
   \param{n}{max possible number of $\lambda_i$}
   \param{k}{effective number of $\lambda_i$}
   \param{h}{step between two successive $\lambda_i$}
   \param{x}{value at which the distribution is evaluated}
   \return{density at $x$}
\end{htmlonly}
\begin{code}

   public static double cdf (int n, int k, double h, double x)\begin{hide} {
      if (x <= 0)
         return 0;
      double r = -Math.expm1(-h*x);
      double u = BetaDist.cdf(k, n - k + 1, r);
      return u;
   }\end{hide}
\end{code}
\begin{tabb} Computes the distribution function $F(x)$, with arguments
  as in the constructor.
 \end{tabb}
\begin{htmlonly}
   \param{n}{max possible number of $\lambda_i$}
   \param{k}{effective number of $\lambda_i$}
   \param{h}{step between two successive $\lambda_i$}
   \param{x}{value at which the distribution is evaluated}
   \return{value of distribution at $x$}
\end{htmlonly}
\begin{code}

   public static double barF (int n, int k, double h, double x)\begin{hide} {
      if (x <= 0)
         return 1.0;
      double r = Math.exp(-h*x);
      double v = BetaDist.cdf(n - k + 1, k, r);
      return v;
   }\end{hide}
\end{code}
\begin{tabb}
  Computes the complementary distribution $\bar F(x)$,
  as in formula (\ref{eq:conv-hypo-equal}).
 \end{tabb}
\begin{htmlonly}
   \param{n}{max possible number of $\lambda_i$}
   \param{k}{effective number of $\lambda_i$}
   \param{h}{step between two successive $\lambda_i$}
   \param{x}{value at which the complementary distribution is evaluated}
   \return{value of complementary distribution at $x$}
\end{htmlonly}
\begin{code}

   public static double inverseF (int n, int k, double h, double u)\begin{hide} {
      if (u < 0.0 || u > 1.0)
          throw new IllegalArgumentException ("u not in [0,1]");
      if (u >= 1.0)
          return Double.POSITIVE_INFINITY;
      if (u <= 0.0)
          return 0.0;

      double z = BetaDist.inverseF(k, n - k + 1, u);
      return -Math.log1p(-z) / h;
   }\end{hide}
\end{code}
\begin{tabb} Computes the inverse distribution $x=F^{-1}(u)$,
 with arguments as in the constructor.
\end{tabb}
\begin{htmlonly}
   \param{n}{max possible number of $\lambda_i$}
   \param{k}{effective number of $\lambda_i$}
   \param{h}{step between two successive $\lambda_i$}
   \param{u}{value at which the inverse distribution is evaluated}
   \return{inverse distribution at $u$}
\end{htmlonly}
\begin{code}

   public double[] getParams()\begin{hide} {
      double[] par = new double[]{m_n, m_k, m_h};
      return par;
   }\end{hide}
\end{code}
\begin{tabb}  Returns the three parameters
of this hypoexponential distribution as array $(n, k, h)$.
\end{tabb}
\begin{htmlonly}
   \return{parameters of the hypoexponential distribution}
\end{htmlonly}
\begin{code}\begin{hide}

   public void setParams (int n, int k, double h) {
      m_n = n;
      m_k = k;
      m_h = h;
      m_lambda = new double[k];
      for(int i = 0; i < k; i++) {
         m_lambda[i] = (n - i)*h;
      }
   }


   public String toString() {
      StringBuilder sb = new StringBuilder();
      Formatter formatter = new Formatter(sb, Locale.US);
      formatter.format(getClass().getSimpleName() + " : params = {" +
           PrintfFormat.NEWLINE);
      formatter.format("   %d, %d, %f", m_n, m_k, m_h);
      formatter.format("}%n");
      return sb.toString();
   }
}\end{hide}
\end{code}
