\defmodule {HypoExponentialDistQuick}

This class is a subclass of \class{HypoExponentialDist}
and also implements  the \emph{hypoexponential} distribution. It uses
different algorithms to compute the probabilities.
The formula (\ref{eq:tail-hypoexp}) for the complementary distribution
is mathematically equivalent to (see
\cite[page 299]{pROS07b} and \cite[Appendix B]{pGER10a})
\begin{equation}
\begin{latexonly}%
\bar F(x) = \mathbb{P}\left[X_1 + \cdots + X_k > x \right]
  = \sum_{i=1}^k e^{-\lambda_i x} \prod_{\substack{j=1\\ j\not=i}}^k \frac{\lambda_j}{\lambda_
j - \lambda_i}.
\label{eq:convolution-hypo}
\end{latexonly}%
\begin{htmlonly}%
\bar F(x) = \mathcal{P}\left[X_1 + \cdots + X_k > x \right]
  = \sum_{i=1}^k e^{-\lambda_i x} \prod_{j=1,\; j\neq i}^k
   {\lambda_j}/{(\lambda_j - \lambda_i)}.
\end{htmlonly}%
\end{equation}


The expression (\ref{eq:convolution-hypo}) is much faster to compute than the
matrix exponential formula (\ref{eq:tail-hypoexp}), but it becomes numerically
unstable when $k$ gets large and/or the differences between the $\lambda_i$
are too small, because it is an alternating sum with relatively large terms
of similar size. When the $\lambda_i$ are close, many of the
factors $\lambda_{j} - \lambda_{i}$ in (\ref{eq:convolution-hypo}) are small,
and the effect of this is amplified when $k$ is large. This gives rise to
large terms of opposite sign in the sum and the formula becomes unstable
due to subtractive cancellation.
%
For example, with the computations done in standard 64-bit floating-point
arithmetic, if the $\lambda_i$ are regularly spaced with differences
of $\lambda_{i+1} - \lambda_{i} = 0.1$ for all $i$, the formula
(\ref{eq:convolution-hypo}) breaks down already for $k \approx 15$, while if
the differences $\lambda_{i+1} - \lambda_{i} = 3$, it gives a few decimal
digits of precision for $k$ up to $\approx 300$.


The formula (\ref{eq:fhypoexp}) for the density is mathematically equivalent to
the much faster formula
\begin{equation}
\begin{latexonly}%
  f(x) = \sum_{i=1}^k\lambda_i e^{-\lambda_i x}
   \prod_{\substack{j=1\\ j\not=i}}^k \frac{\lambda_j}{\lambda_j - \lambda_i},
\eqlabel{eq:fhypoexp2}
\end{latexonly}%
\begin{htmlonly}%
 f(x) = \sum_{i=1}^k\lambda_i e^{-\lambda_i x}
   \prod_{j=1,\; j\neq i}^k {\lambda_j}/{(\lambda_j - \lambda_i)},
\end{htmlonly}%
\end{equation}
which is also  numerically
unstable when $k$ gets large and/or the differences between the $\lambda_i$
are too small.


\bigskip\hrule

%%%%%%%%%%%%%%%%%%%%%%%%%%%%%%%%%%%%%%%%
\begin{code}\begin{hide}
/*
 * Class:        HypoExponentialDistQuick
 * Description:  Hypo-exponential distribution
 * Environment:  Java
 * Software:     SSJ
 * Copyright (C) 2001  Pierre L'Ecuyer and Universite de Montreal
 * Organization: DIRO, Universite de Montreal
 * @author       Richard Simard
 * @since        January 2011

 * SSJ is free software: you can redistribute it and/or modify it under
 * the terms of the GNU General Public License (GPL) as published by the
 * Free Software Foundation, either version 3 of the License, or
 * any later version.

 * SSJ is distributed in the hope that it will be useful,
 * but WITHOUT ANY WARRANTY; without even the implied warranty of
 * MERCHANTABILITY or FITNESS FOR A PARTICULAR PURPOSE.  See the
 * GNU General Public License for more details.

 * A copy of the GNU General Public License is available at
   <a href="http://www.gnu.org/licenses">GPL licence site</a>.
 */
\end{hide}
package umontreal.iro.lecuyer.probdist;
\begin{hide}
import java.util.Formatter;
import java.util.Locale;
import umontreal.iro.lecuyer.util.*;
import umontreal.iro.lecuyer.functions.MathFunction;\end{hide}

public class HypoExponentialDistQuick extends HypoExponentialDist\begin{hide} {
   private double[] m_H;

   private static double[] computeH (double[] lambda) {
      int k = lambda.length;
      double[] H = new double[k];
      double tem;
      int j;
      for (int i = 0; i < k; i++) {
         tem = 1.0;
         for (j = 0; j < i; j++)
            tem *= lambda[j] / (lambda[j] - lambda[i]);
         for (j = i + 1; j < k; j++)
            tem *= lambda[j] / (lambda[j] - lambda[i]);
         H[i] = tem;
      }
      return H;
   }


   private static double m_density(double[] lambda, double[] H, double x) {
      if (x < 0)
         return 0;

      int k = lambda.length;
      double tem;
      double prob = 0;
      for (int j = 0; j < k; j++) {
         tem = Math.exp (-lambda[j] * x);
         if (tem > 0)
            prob += lambda[j] * H[j] * tem;
      }

      return prob;
   }


   private static double m_barF(double[] lambda, double[] H, double x) {
      if (x <= 0.)
         return 1.;

      int k = lambda.length;
      double tem;
      double prob = 0;            // probability
      for (int j = 0; j < k; j++) {
         tem = Math.exp (-lambda[j] * x);
         if (tem > 0)
            prob += H[j] * tem;
      }
      return prob;
	}


   private static double m_cdf(double[] lambda, double[] H, double x) {
      if (x <= 0.)
         return 0.;

      int k = lambda.length;
      double tem = Math.exp(-lambda[0] * x);
      if (tem <= 0)
         return 1.0 - HypoExponentialDistQuick.m_barF(lambda, H, x);

      double prob = 0;            // cumulative probability
      for (int j = 0; j < k; j++) {
         tem = Math.expm1 (-lambda[j] * x);
         prob += H[j] * tem;
      }
      return -prob;
   }


   private static class myFunc implements MathFunction {
      // For inverseF
      private double[] m_lam;
      private double m_u;

      public myFunc (double[] lam, double u) {
         m_lam = lam;
         m_u = u;
      }

      public double evaluate (double x) {
         return m_u - HypoExponentialDistQuick.cdf(m_lam, x);
      }
   }
\end{hide}
\end{code}
%%%%%%%%%%%%%%%%%%%%%%%%%%%%%%%%%%%%%%%%%
\subsubsection* {Constructor}

\begin{code}

   public HypoExponentialDistQuick (double[] lambda)\begin{hide} {
      super(lambda);
   }\end{hide}
\end{code}
  \begin{tabb} Constructs a \texttt{HypoExponentialDistQuick} object,
with rates $\lambda_i = $ \texttt{lambda[$i-1$]}, $i = 1,\ldots,k$.
  \end{tabb}
\begin{htmlonly}
   \param{lambda}{rates of the hypoexponential distribution}
\end{htmlonly}


%%%%%%%%%%%%%%%%%%%%%%%%%%%%%%%%%%%
\subsubsection* {Methods}

\begin{code}\begin{hide}
  /*  These public methods are necessary so that methods cdf,
   *  barF and inverseF used are those of the present
   *  class and not those of the mother class.
   */
   public double density (double x) {
     return m_density (m_lambda, m_H, x);
   }

   public double cdf (double x) {
      return m_cdf (m_lambda, m_H, x);
   }

   public double barF (double x) {
      return m_barF (m_lambda, m_H, x);
   }

   public double inverseF (double u) {
      return m_inverseF (m_lambda, m_H, u);
   }\end{hide}

   public static double density (double[] lambda, double x)\begin{hide} {
      testLambda (lambda);
      double[] H = computeH (lambda);
      return m_density(lambda, H, x);
   }\end{hide}
\end{code}
  \begin{tabb} Computes the density function $f(x)$, with $\lambda_i = $
\texttt{lambda[$i-1$]}, $i = 1,\ldots,k$.
\end{tabb}
\begin{htmlonly}
   \param{lambda}{rates of the hypoexponential distribution}
   \param{x}{value at which the density is evaluated}
   \return{density at $x$}
\end{htmlonly}
\begin{code}

   public static double cdf (double[] lambda, double x)\begin{hide} {
      testLambda (lambda);
      double[] H = computeH (lambda);
      return m_cdf(lambda, H, x);
   }\end{hide}
\end{code}
\begin{tabb}  Computes the distribution function $F(x)$,
with $\lambda_i = $ \texttt{lambda[$i-1$]}, $i = 1,\ldots,k$.
 \end{tabb}
\begin{htmlonly}
   \param{lambda}{rates of the hypoexponential distribution}
   \param{x}{value at which the distribution is evaluated}
   \return{value of distribution at $x$}
\end{htmlonly}
\begin{code}

   public static double barF (double[] lambda, double x)\begin{hide} {
      testLambda (lambda);
      double[] H = computeH (lambda);
      return m_barF(lambda, H, x);
   }\end{hide}
\end{code}
\begin{tabb}
  Computes the complementary distribution $\bar F(x)$,
with $\lambda_i = $ \texttt{lambda[$i-1$]}, $i = 1,\ldots,k$.
 \end{tabb}
\begin{htmlonly}
   \param{lambda}{rates of the hypoexponential distribution}
   \param{x}{value at which the complementary distribution is evaluated}
   \return{value of complementary distribution at $x$}
\end{htmlonly}
\begin{code}

   public static double inverseF (double[] lambda, double u)\begin{hide} {
      testLambda (lambda);
      double[] H = computeH (lambda);
      return m_inverseF(lambda, H, u);
   }

   private static double m_inverseF(double[] lambda, double[] H, double u) {
      if (u < 0.0 || u > 1.0)
          throw new IllegalArgumentException ("u not in [0,1]");
      if (u >= 1.0)
          return Double.POSITIVE_INFINITY;
      if (u <= 0.0)
          return 0.0;

      final double EPS = 1.0e-12;
      myFunc fonc = new myFunc (lambda, u);
      double x1 = getMean (lambda);
      double v = m_cdf(lambda, H, x1);
      if (u <= v)
         return RootFinder.brentDekker (0, x1, fonc, EPS);

      // u > v
      double x2 = 4.0*x1 + 1.0;
      v = m_cdf(lambda, H, x2);
      while (v < u) {
         x1 = x2;
         x2 = 4.0*x2;
         v = m_cdf(lambda, H, x2);
      }
      return RootFinder.brentDekker (x1, x2, fonc, EPS);
   }\end{hide}
\end{code}
\begin{tabb} Computes the inverse distribution function $F^{-1}(u)$,
with $\lambda_i = $ \texttt{lambda[$i-1$]}, $i = 1,\ldots,k$.
% It uses a root-finding method and is very slow.
\end{tabb}
\begin{htmlonly}
   \param{lambda}{rates of the hypoexponential distribution}
   \param{u}{value at which the inverse distribution is evaluated}
   \return{inverse distribution at $u$}
\end{htmlonly}
\begin{code}\begin{hide}

   public void setLambda (double[] lambda) {
      if (lambda == null)
         return;
      super.setLambda (lambda);
      m_H = computeH (lambda);
   }

   public String toString() {
      StringBuilder sb = new StringBuilder();
      Formatter formatter = new Formatter(sb, Locale.US);
      formatter.format(getClass().getSimpleName() + " : lambda = {" +
           PrintfFormat.NEWLINE);
      int k = m_lambda.length;
      for(int i = 0; i < k; i++) {
         formatter.format("   %f%n", m_lambda[i]);
      }
      formatter.format("}%n");
      return sb.toString();
   }
}\end{hide}
\end{code}
