\defmodule{Pearson5Dist}

\textbf{THIS CLASS HAS BEEN RENAMED \class{InverseGammaDist}}.

Extends the class \class{ContinuousDistribution} for
the {\em Pearson type V\/} distribution with shape parameter
$\alpha > 0$ and scale parameter $\beta > 0$.
The density function is given by
\begin{htmlonly}
\eq
  f(x) = (\beta^{\alpha}\exp^{-\beta / x}) / (x^{\alpha+1} \Gamma(\alpha))
  \qquad \mbox{for } x > 0,
\endeq
 and $f(x) = 0$ otherwise,
\end{htmlonly}
\begin{latexonly}
\eq
  f(x) = \left\{\begin{array}{ll} \displaystyle
        \frac{\beta^{\alpha}e^{-\beta / x}}{x^{\alpha+ 1} \Gamma(\alpha)}
   & \quad \mbox{for } x > 0 \\[12pt]
   0  & \quad \mbox{otherwise,}
   \end{array} \right.
  \eqlabel{eq:fpearson5}
\endeq
\end{latexonly}
where $\Gamma$ is the gamma function.
The distribution function is given by
\begin{htmlonly}
\eq
   F(x) = 1 - F_{G}(1 / x),
   \quad \mbox{for } x > 0,
\endeq
\end{htmlonly}
\begin{latexonly}
\eq
   F(x) = 1 - F_{G}\left(\frac{1}{x}\right)
   \qquad \mbox{for } x > 0,
   \eqlabel{eq:Fpearson5}
\endeq
\end{latexonly}
 and $F(x) = 0$ otherwise, where $F_{G}(x)$ is the distribution function
of a gamma
distribution with shape parameter $\alpha$ and scale parameter $\beta$.

\bigskip\hrule

%%%%%%%%%%%%%%%%%%%%%%%%%%%%%%%%%%%%%%%%
\begin{code}
\begin{hide}
/*
 * Class:        Pearson5Dist
 * Description:  Pearson type V distribution
 * Environment:  Java
 * Software:     SSJ 
 * Copyright (C) 2001  Pierre L'Ecuyer and Universite de Montreal
 * Organization: DIRO, Universite de Montreal
 * @author       
 * @since

 * SSJ is free software: you can redistribute it and/or modify it under
 * the terms of the GNU General Public License (GPL) as published by the
 * Free Software Foundation, either version 3 of the License, or
 * any later version.

 * SSJ is distributed in the hope that it will be useful,
 * but WITHOUT ANY WARRANTY; without even the implied warranty of
 * MERCHANTABILITY or FITNESS FOR A PARTICULAR PURPOSE.  See the
 * GNU General Public License for more details.

 * A copy of the GNU General Public License is available at
   <a href="http://www.gnu.org/licenses">GPL licence site</a>.
 */
\end{hide}
package umontreal.iro.lecuyer.probdist;
\begin{hide}
import umontreal.iro.lecuyer.util.Num;
\end{hide}
@Deprecated
public class Pearson5Dist extends ContinuousDistribution\begin{hide} {
   protected double alpha;
   protected double beta;
   protected double logam;   // Ln (Gamma(alpha))

\end{hide}\end{code}
%%%%%%%%%%%%%%%%%%%%%%%%%%%%%%%%%%%%%%%%%
\subsubsection* {Constructor}

\begin{code}

   public Pearson5Dist (double alpha, double beta)\begin{hide} {
      setParam (alpha, beta);
   }\end{hide}
\end{code}
\begin{tabb}
\textbf{THIS CLASS HAS BEEN RENAMED \class{InverseGammaDist}}.
   Constructs a \texttt{Pearson5Dist} object with parameters
   $\alpha$ = \texttt{alpha} and $\beta$ = \texttt{beta}.
\end{tabb}

%%%%%%%%%%%%%%%%%%%%%%%%%%%%%%%%%%%
\subsubsection* {Methods}
\begin{code}\begin{hide}

   public double density (double x) {
      if (x <= 0.0)
         return 0.0;
      return Math.exp (alpha * Math.log (beta/x) - (beta / x) - logam) / x;
   }

   public double cdf (double x) {
      return cdf (alpha, beta, x);
   }

   public double barF (double x) {
      return barF (alpha, beta, x);
   }

   public double inverseF (double u) {
      return inverseF (alpha, beta, u);
   }

   public double getMean () {
      return getMean (alpha, beta);
   }

   public double getVariance () {
      return getVariance (alpha, beta);
   }

   public double getStandardDeviation () {
      return getStandardDeviation (alpha, beta);
   }\end{hide}

   public static double density (double alpha, double beta, double x)\begin{hide} {
      if (alpha <= 0.0)
         throw new IllegalArgumentException("alpha <= 0");
      if (beta <= 0.0)
         throw new IllegalArgumentException("beta <= 0");
      if (x <= 0.0)
         return 0.0;

      return Math.exp (alpha * Math.log (beta/x) - (beta / x) - Num.lnGamma (alpha)) / x;
   }\end{hide}
\end{code}
\begin{tabb}
   Computes the density function of a Pearson V distribution with shape
parameter $\alpha$
   and scale parameter $\beta$.
\end{tabb}
\begin{code}

   public static double cdf (double alpha, double beta, double x)\begin{hide} {
      if (alpha <= 0.0)
         throw new IllegalArgumentException("alpha <= 0");
      if (beta <= 0.0)
         throw new IllegalArgumentException("beta <= 0");
      if (x <= 0.0)
         return 0.0;

      return GammaDist.barF (alpha, beta, 15, 1.0 / x);
   }\end{hide}
\end{code}
\begin{tabb}
   Computes the density function of a Pearson V distribution with shape
parameter $\alpha$
   and scale parameter $\beta$.
\end{tabb}
\begin{code}

   public static double barF (double alpha, double beta, double x)\begin{hide} {
      if (alpha <= 0.0)
         throw new IllegalArgumentException("alpha <= 0");
      if (beta <= 0.0)
         throw new IllegalArgumentException("beta <= 0");
      if (x <= 0.0)
         return 1.0;

      return GammaDist.cdf (alpha, beta, 15, 1.0 / x);
   }\end{hide}
\end{code}
\begin{tabb}
   Computes the complementary distribution function of a Pearson V distribution
   with shape parameter $\alpha$ and scale parameter $\beta$.
\end{tabb}
\begin{code}

   public static double inverseF (double alpha, double beta, double u)\begin{hide} {
      if (alpha <= 0.0)
         throw new IllegalArgumentException("alpha <= 0");
      if (beta <= 0.0)
         throw new IllegalArgumentException("beta <= 0");

      return 1.0 / GammaDist.inverseF (alpha, beta, 15, 1 - u);
   }\end{hide}
\end{code}
\begin{tabb}
   Computes the inverse distribution function of a Pearson V distribution
   with shape parameter $\alpha$ and scale parameter $\beta$.
\end{tabb}
\begin{code}

   public static double[] getMLE (double[] x, int n)\begin{hide} {
      double[] y = new double[n];

      for (int i = 0; i < n; i++) {
	 if(x[i] > 0)
	     y[i] = 1.0 / x[i];
	 else
	     y[i] = 1.0E100;
      }

      return GammaDist.getMLE (y, n);
   }\end{hide}
\end{code}
\begin{tabb}
   Estimates the parameters $(\alpha,\beta)$ of the Pearson V distribution
   using the maximum likelihood method, from the $n$ observations
   $x[i]$, $i = 0, 1,\ldots, n-1$. The estimates are returned in a two-element
    array, in regular order: [$\alpha$, $\beta$].
   \begin{detailed}
   The equations of the maximum likelihood are the same as the equations of
   the gamma distribution, with the sample $y_i = 1/x_i$.
 \end{detailed}
\end{tabb}
\begin{htmlonly}
   \param{x}{the list of observations to use to evaluate parameters}
   \param{n}{the number of observations to use to evaluate parameters}
   \return{returns the parameters [$\hat{\alpha}, \hat{\beta}$]}
\end{htmlonly}
\begin{code}

   public static Pearson5Dist getInstanceFromMLE (double[] x, int n)\begin{hide} {
      double parameters[] = getMLE (x, n);
      return new Pearson5Dist (parameters[0], parameters[1]);
   }\end{hide}
\end{code}
\begin{tabb}
   Creates a new instance of a Pearson V distribution with parameters $\alpha$
   and $\beta$ estimated using the maximum likelihood method based on the $n$
   observations $x[i]$, $i = 0, 1, \ldots, n-1$.
\end{tabb}
\begin{htmlonly}
   \param{x}{the list of observations to use to evaluate parameters}
   \param{n}{the number of observations to use to evaluate parameters}
\end{htmlonly}
\begin{code}

   public static double getMean (double alpha, double beta)\begin{hide} {
      if (alpha <= 0.0)
         throw new IllegalArgumentException("alpha <= 0");
      if (beta <= 0.0)
         throw new IllegalArgumentException("beta <= 0");

      return (beta / (alpha - 1.0));
   }\end{hide}
\end{code}
\begin{tabb}
   Computes and returns the mean $E[X] = \beta / (\alpha - 1)$ of a Pearson V
   distribution with shape parameter $\alpha$ and scale parameter $\beta$.
\end{tabb}
\begin{code}

   public static double getVariance (double alpha, double beta)\begin{hide} {
      if (alpha <= 0.0)
         throw new IllegalArgumentException("alpha <= 0");
      if (beta <= 0.0)
         throw new IllegalArgumentException("beta <= 0");

      return ((beta * beta) / ((alpha - 1.0) * (alpha - 1.0) * (alpha - 2.0)));
   }\end{hide}
\end{code}
\begin{tabb}
   Computes and returns the variance $\mbox{Var}[X] = \beta^2 / ((\alpha -
1)^2(\alpha - 2)$
   of a Pearson V distribution with shape parameter $\alpha$ and scale
parameter $\beta$.
\end{tabb}
\begin{code}

   public static double getStandardDeviation (double alpha, double beta)\begin{hide} {
      return Math.sqrt (getVariance (alpha, beta));
   }\end{hide}
\end{code}
\begin{tabb}
   Computes and returns the standard deviation of a Pearson V distribution with
   shape parameter $\alpha$ and scale parameter $\beta$.
\end{tabb}
\begin{code}

   public double getAlpha()\begin{hide} {
      return alpha;
   }\end{hide}
\end{code}
\begin{tabb}
   Returns the $\alpha$ parameter of this object.
\end{tabb}
\begin{code}

   public double getBeta()\begin{hide} {
      return beta;
   }\end{hide}
\end{code}
\begin{tabb}
   Returns the $\beta$ parameter of this object.
\end{tabb}
\begin{code}

   public void setParam (double alpha, double beta)\begin{hide} {
      if (alpha <= 0.0)
         throw new IllegalArgumentException("alpha <= 0");
      if (beta <= 0.0)
         throw new IllegalArgumentException("beta <= 0");
      supportA = 0.0;
      this.alpha = alpha;
      this.beta = beta;
      logam = Num.lnGamma (alpha);
   }\end{hide}
\end{code}
\begin{tabb}
   Sets the parameters $\alpha$ and $\beta$ of this object.
\end{tabb}
\begin{code}

   public double[] getParams ()\begin{hide} {
      double[] retour = {alpha, beta};
      return retour;
   }\end{hide}
\end{code}
\begin{tabb}
   Return a table containing the parameters of the current distribution.
   This table is put in regular order: [$\alpha$, $\beta$].
\end{tabb}
\begin{hide}\begin{code}

   public String toString ()\begin{hide} {
      return getClass().getSimpleName() + " : alpha = " + alpha + ", beta = " + beta;
   }\end{hide}
\end{code}
\begin{tabb}
   Returns a \texttt{String} containing information about the current distribution.
\end{tabb}\end{hide}
\begin{code}\begin{hide}
}\end{hide}
\end{code}
