\defmodule {StudentDist}

Extends the class \class{ContinuousDistribution} for
the \emph{Student} $t$-distribution \cite[page 362]{tJOH95b}
with $n$ degrees of freedom, where $n$ is a positive integer.
Its density is
\begin{htmlonly}
\eq
        f (x) = [\Gamma((n + 1)/2)/
                        (\Gamma (n/2) \sqrt{\pi n})]
            [1 + x^2/n]^{-(n+1)/2}
            \qquad\mbox {for } -\infty < x < \infty,
\endeq
\end{htmlonly}
\begin{latexonly}
\eq
        f (x) = \frac{\Gamma\left ((n + 1)/2 \right)}
                        {\Gamma (n/2) \sqrt{\pi n}}
            \left(1 + \frac{x^2}{n}\right)^{-(n+1)/2}
            \qquad\mbox {for } -\infty < x < \infty,
                                          \eqlabel{eq:fstudent}
\endeq
\end{latexonly}
where $\Gamma(x)$ is the gamma function defined in
\latex{(\ref{eq:Gamma})}\html{\class{GammaDist}}.

\bigskip\hrule\bigskip

%%%%%%%%%%%%%%%%%%%%%%%%%%%%%%%%%%%%%%%%
\begin{code}\begin{hide}
/*
 * Class:        StudentDistDist
 * Description:  Student-t distribution
 * Environment:  Java
 * Software:     SSJ
 * Copyright (C) 2001  Pierre L'Ecuyer and Universite de Montreal
 * Organization: DIRO, Universite de Montreal
 * @author       Richard Simard
 * @since        March 2009

 * SSJ is free software: you can redistribute it and/or modify it under
 * the terms of the GNU General Public License (GPL) as published by the
 * Free Software Foundation, either version 3 of the License, or
 * any later version.

 * SSJ is distributed in the hope that it will be useful,
 * but WITHOUT ANY WARRANTY; without even the implied warranty of
 * MERCHANTABILITY or FITNESS FOR A PARTICULAR PURPOSE.  See the
 * GNU General Public License for more details.

 * A copy of the GNU General Public License is available at
   <a href="http://www.gnu.org/licenses">GPL licence site</a>.
 */
\end{hide}
package umontreal.iro.lecuyer.probdist;
\begin{hide}
import umontreal.iro.lecuyer.util.Num;
import umontreal.iro.lecuyer.functions.MathFunction;
\end{hide}

public class StudentDist extends ContinuousDistribution\begin{hide} {
   protected int n;
   private double factor;
   private static final int NLIM1 = 100000;
/*
   private static double cdfPeizer (int n, double x) {
      // Peizer-Pratt normal approximation for the cdf (n, u)
      // \cite{tPEI68a}
      double v = Math.log1p(x*x/n) / (n - 5.0/6.0);
      double z = -(n - 2.0/3.0 + 0.1/n) * Math.sqrt(v);
      double u = NormalDist.cdf01 (z);
      if (x >= 0.0)
         return 1.0 - u;
      return u;
   }

   private static double invPeizer (int n, double u) {
      // Peizer-Pratt normal approximation for the inverseF (n, u)
      // \cite{tPEI68a}
      double z = NormalDist.inverseF01 (u);
      double q = z / (n - 2.0/3.0 + 0.1/n);
      double v = q*q*(n - 5.0/6.0);
      double t = Math.sqrt(n * Math.expm1(v));
      if (u >= 0.5)
         return t;
      else
         return -t;
   }
*/

   private static double cdfGaver (int n, double x) {
      // Gaver-Kafadar normal approximation for the cdf
      // \cite{tGAV84a}
      double v = Math.log1p(x * x / n) / (n - 1.5);
      double z = -(n - 1) * Math.sqrt(v);
      double u = NormalDist.cdf01 (z);
      if (x >= 0.0)
         return 1.0 - u;
      return u;
   }


   private static double invGaver (int n, double u) {
      // Gaver-Kafadar normal approximation for the inverse
      // \cite{tGAV84a}
      double z = NormalDist.inverseF01 (u);
      double q = z / (n - 1.0);
      double v = q * q * (n - 1.5);
      double t = Math.sqrt(n * Math.expm1(v));
      if (u >= 0.5)
         return t;
      else
         return -t;
   }


   private static class Function implements MathFunction {
      private int n;
      private double[] xi;

      public Function (double[] x, int n) {
         this.n = n;
         this.xi = new double[n];
         System.arraycopy (x, 0, this.xi, 0, n);
      }

      public double evaluate (double x) {
         if (x <= 0.0)
            return 1e200;
         double sum = 0.0;
         for (int i = 0; i < n; i++)
            sum += Math.log (density ((int) Math.round (x), xi[i]));
         return sum;
      }
   }
\end{hide}\end{code}

%%%%%%%%%%%%%%%%%%%%%%%%%%%%%%%%%%%%%%%%%
\subsubsection* {Constructors}

\begin{code}

   public StudentDist (int n)\begin{hide} {
     setN (n);
   }\end{hide}
\end{code}
\begin{tabb} Constructs a \texttt{StudentDist} object with \texttt{n} degrees of freedom.
\end{tabb}

%%%%%%%%%%%%%%%%%%%%%%%%%%%%%%%%%%%
\subsubsection* {Methods}

\begin{code}\begin{hide}

   public double density (double x) {
      return factor*Math.pow (1.0 / (1.0 + x*x/n), (n + 1)/2.0);
   }

   public double cdf (double x) {
      return cdf (n, x);
   }

   public double barF (double x) {
      return barF (n, x);
   }

   public double inverseF (double u) {
      return inverseF (n, u);
   }

   public double getMean() {
      return StudentDist.getMean (n);
   }

   public double getVariance() {
      return StudentDist.getVariance (n);
   }

   public double getStandardDeviation() {
      return StudentDist.getStandardDeviation (n);
   }\end{hide}

   public static double density (int n, double x)\begin{hide} {
      double factor = Num.gammaRatioHalf(n/2.0)/ Math.sqrt (n*Math.PI);
      return factor*Math.pow (1.0 / (1.0 + x*x/n), (n + 1)/2.0);
   }\end{hide}
\end{code}
\begin{tabb} Computes the density function (\ref{eq:fstudent}) of a
   Student $t$-distribution with $n$ degrees of freedom.
\end{tabb}
\begin{code}

   public static double cdf (int n, double x)\begin{hide} {
      if (n < 1)
        throw new IllegalArgumentException ("n < 1");
      if (n == 1)
         return CauchyDist.cdf(0, 1, x);

      if (x > 1.0e10)
         return 1.0;
      if (n > NLIM1)
         return cdfGaver(n, x);

      double r = Math.abs(x);
      if (r < 1.0e20)
         r = Math.sqrt (n + x*x);

      double z;
      if (x >= 0.0)
         z = 0.5*(1.0 + x/r);
      else
         z = 0.5*n/(r*(r - x));

      if (n == 2)
         return z;
      return BetaSymmetricalDist.cdf (0.5*n, 15, z);
   }\end{hide}
\end{code}
  \begin{tabb}
  Computes the Student $t$-distribution function $u=F(x)$ with $n$ degrees of freedom.
  Gives 13 decimal digits of precision for $n \le 10^5$.
 For $n >  10^5$, gives  at least 6 decimal digits of precision everywhere, and
 at least 9 decimal digits of precision for all $u >  10^{-15}$.
  \end{tabb}
\begin{code}

   @Deprecated
   public static double cdf2 (int n, int d, double x)\begin{hide} {
      if (d <= 0)
         throw new IllegalArgumentException ("student2:   d <= 0");
      return cdf (n, x);
   }\end{hide}
\end{code}
  \begin{tabb}
Same as \method{cdf}{int,double}\texttt{(n, x)}.
  \end{tabb}
\begin{code}

   public static double barF (int n, double x)\begin{hide} {
      if (n < 1)
        throw new IllegalArgumentException ("n < 1");
      if (n == 1)
         return CauchyDist.barF(0, 1, x);

      if (n == 2) {
         double z = Math.abs(x);
         if (z < 1.0e20)
            z = Math.sqrt(2.0 + x*x);
         if (x <= 0.) {
            if (x < -1.0e10)
               return 1.0;
            return 0.5* (1.0 - x / z);
         } else
            return 1.0 / (z * (z + x));
      }

      return cdf (n, -x);
   }\end{hide}
\end{code}
\begin{tabb} Computes the complementary distribution function $v = \bar{F}(x)$
with $n$ degrees of freedom. Gives 13 decimal digits of precision for $n \le 10^5$.
 For $n >  10^5$, gives  at least 6 decimal digits of precision everywhere, and
 at least 9 decimal digits of precision for all $v >  10^{-15}$.
\end{tabb}
\begin{code}

   public static double inverseF (int n, double u)\begin{hide} {
        if (n < 1)
            throw new IllegalArgumentException ("Student:   n < 1");
        if (u > 1.0 || u < 0.0)
            throw new IllegalArgumentException ("Student:   u not in [0, 1]");
        if (u <= 0.0)
           return Double.NEGATIVE_INFINITY;
        if (u >= 1.0)
           return Double.POSITIVE_INFINITY;

        if (1 == n)
           return CauchyDist.inverseF(0, 1, u);

        if (2 == n)
           return (2.0*u - 1.0) / Math.sqrt(2.0*u*(1.0 - u));

        if (n > NLIM1)
           return invGaver(n, u);
        double z = BetaSymmetricalDist.inverseF (0.5*n, u);
        return (z - 0.5) * Math.sqrt(n / (z*(1.0 - z)));
   }\end{hide}
\end{code}
  \begin{tabb}
  Returns the inverse $x = F^{-1}(u)$ of
  Student $t$-distribution function with $n$ degrees of freedom.
  Gives 13 decimal digits of precision for $n \le 10^5$,
  and at least 9 decimal digits of precision for $n >  10^5$.
  \end{tabb}
\begin{code}

   public static double[] getMLE (double[] x, int m)\begin{hide} {
      double sum = 0.0;
      double[] parameters = new double[1];

      if (m <= 0)
         throw new IllegalArgumentException ("m <= 0");

      double var = 0.0;
      for (int i = 0; i < m; i++)
         var += x[i] * x[i];
      var /= (double) m;

      Function f = new Function (x, m);

      double n0 = Math.round ((2.0 * var) / (var - 1.0));
      double fn0 = f.evaluate (n0);
      double min = fn0;
      double fn1 = f.evaluate (n0 + 1.0);
      double fn_1 = f.evaluate (n0 - 1.0);

      parameters[0] = n0;

      if (fn_1 > fn0) {
         double n = n0 - 1.0;
         double y;
         while (((y = f.evaluate (n)) > min) && (n >= 1.0)) {
            min = y;
            parameters[0] = n;
            n -= 1.0;
         }

      } else if (fn1 > fn0) {
         double n = n0 + 1.0;
         double y;
         while ((y = f.evaluate (n)) > min) {
            min = y;
            parameters[0] = n;
            n += 1.0;
         }
      }
      return parameters;
   }\end{hide}
\end{code}
\begin{tabb}
   Estimates the parameter $n$ of the Student $t$-distribution
   using the maximum likelihood method, from the $m$ observations
   $x[i]$, $i = 0, 1,\ldots, m-1$. The estimate is returned in a one-element
   array.
\end{tabb}
\begin{htmlonly}
   \param{x}{the list of observations to use to evaluate parameters}
   \param{m}{the number of observations to use to evaluate parameters}
   \return{returns the parameter [$\hat{n}$]}
\end{htmlonly}
\begin{code}

   public static StudentDist getInstanceFromMLE (double[] x, int m)\begin{hide} {
      double parameters[] = getMLE (x, m);
      return new StudentDist ((int) parameters[0]);
   }\end{hide}
\end{code}
\begin{tabb}
   Creates a new instance of a Student $t$-distribution with parameter $n$
   estimated using the maximum likelihood method based on the $m$ observations
   $x[i]$, $i = 0, 1, \ldots, m-1$.
\end{tabb}
\begin{htmlonly}
   \param{x}{the list of observations to use to evaluate parameters}
   \param{m}{the number of observations to use to evaluate parameters}
\end{htmlonly}
\begin{code}

   public static double getMean (int n)\begin{hide} {
     if (n < 2)
        throw new IllegalArgumentException ("n <= 1");
      return 0;
   }\end{hide}
\end{code}
\begin{tabb} Returns the mean $E[X] = 0$ of the Student $t$-distribution with
 parameter $n$.
\end{tabb}
\begin{htmlonly}
   \return{the mean of the Student $t$-distribution $E[X] = 0$}
\end{htmlonly}
\begin{code}

   public static double getVariance (int n)\begin{hide} {
      if (n < 3)
         throw new IllegalArgumentException("n <= 2");
      return (n / (n - 2.0));
   }\end{hide}
\end{code}
\begin{tabb}  Computes and returns the variance $\mbox{Var}[X] = n/(n - 2)$
   of the Student $t$-distribution with parameter $n$.
\end{tabb}
\begin{htmlonly}
   \return{the variance of the Student $t$-distribution
    $\mbox{Var}[X] = n / (n - 2)$}
\end{htmlonly}
\begin{code}

   public static double getStandardDeviation (int n)\begin{hide} {
      return Math.sqrt (StudentDist.getVariance (n));
   }\end{hide}
\end{code}
\begin{tabb}  Computes and returns the standard deviation
   of the Student $t$-distribution with parameter $n$.
\end{tabb}
\begin{htmlonly}
   \return{the standard deviation of the Student $t$-distribution}
\end{htmlonly}
\begin{code}

   public int getN()\begin{hide} {
      return n;
   }\end{hide}
\end{code}
  \begin{tabb} Returns the parameter $n$ associated with this object.
  \end{tabb}
\begin{code}

   public void setN (int n)\begin{hide} {
     if (n <= 0)
        throw new IllegalArgumentException ("n <= 0");
      this.n = n;
      factor = Num.gammaRatioHalf(n/2.0) / Math.sqrt (n*Math.PI);
   }\end{hide}
\end{code}
  \begin{tabb} Sets the parameter $n$ associated with this object.
  \end{tabb}
\begin{code}

   public double[] getParams ()\begin{hide} {
      double[] retour = {n};
      return retour;
   }\end{hide}
\end{code}
\begin{tabb}
   Return a table containing the parameter of the current distribution.
\end{tabb}
\begin{hide}\begin{code}

   public String toString ()\begin{hide} {
      return getClass().getSimpleName() + " : n = " + n;
   }\end{hide}
\end{code}
\begin{tabb}
   Returns a \texttt{String} containing information about the current distribution.
\end{tabb}\end{hide}
\begin{code}\begin{hide}
}\end{hide}
\end{code}
